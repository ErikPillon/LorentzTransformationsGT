\section{Introduction}
%In each reference frame, an observer can use a local coordinate system (most exclusively Cartesian coordinates in this context) to measure lengths, and a clock to measure time intervals. An observer is a real or imaginary entity that can take measurements, say humans, or any other living organism—or even robots and computers. An event is something that happens at a point in space at an instant of time, or more formally a point in spacetime. The transformations connect the space and time coordinates of an event as measured by an observer in each frame.

Traditionally, two postulates are put at the beginning of Special Relativity, from
which all other results can be derived:
\begin{enumerate}[(i)]
	\item The Principle of Relativity
	\item The constancy of the speed of light
\end{enumerate}
From these principles the Lorentz transformation may be derived in numerous ways. This approach does not concentrate on a single Lorentz transformation but works with the totality of all transformations admitted by the principle of relativity. We therefore here set out to derive the Lorentz transformation in a manner that takes into account this central role of principle $ (i) $, and take $ (ii) $ only to decide between the numbers $ -1, 0 $, and $ 1 $.
