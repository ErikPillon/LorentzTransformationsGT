\section{Invariance of the Speed of Light.}
The yet undetermined constant $ K $ has the physical dimension of reciprocal velocity squared. To interpret it we remark that the foundamental relation holds
\[(dx^0)^2+K(d\mathbf{x})^2=(dx^{\bar{0}})^2+K(d\mathbf{\bar{x}})^2 \]
As a consequence, for any motion $ \mathbf{x}=\mathbf{x}(x^0) $ satisfying $ (d\mathbf{x}/dx^O)^2=-1/K $ in one inertial system the analogous relation is true in any other inertial system. Therefore, $ c:= 1/\sqrt{-K} $ plays the role of a uniquely determined \emph{invariant speed}.\footnote{It is an experimental question whether such exists in nature, and if so, what is its value.}

In what follows, we shall most of the time assume performed the rescaling indicated above, and use units where $ c = 1 $-i.e., speeds are expressed as multiples of $ c $.

Then we have
\begin{equation}
K=-1,\qquad a(v)=\frac{1}{\sqrt{1-v^2}}=:\gamma,\qquad \frac{\gamma-1}{v^2}\equiv \frac{\gamma^2}{\gamma+1},
\end{equation}
and eq.~\eqref{eq:semifinal_formulation} becomes the (special) Lorentz transformation ('\emph{Lorentz boost}')
\begin{equation}
\begin{cases}
x^{\bar{0}}=\gamma(x^0-\mathbf{v}\mathbf{x})\\
\bar{\mathbf{x}}=\mathbf{x}+\frac{\gamma^2}{\gamma+1}\mathbf{v}(\mathbf{v}\mathbf{x})-\gamma\mathbf{v}x^0.
\end{cases}
\end{equation}
By composing space-time translations, space rotations and Lorentz boosts in various ways we get more complicated transformations. Homogeneous ones will be called \emph{(general) Lorentz transformations}, inhomogeneous ones will be called \emph{Poincar\'e transformations}.






