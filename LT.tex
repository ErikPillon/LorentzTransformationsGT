\section{The Lorentz Transformations}
Consider a number of labs in free flight. Such a lab defines
an inertial system $ I $. Each (pointlike) event may be recorded by noting its coordinates $ x, y, z $ with respect to a rectangular Cartesian coordinate system anchored in $ I $ together with the reading $ t $ of a clock attached to $ 1 $. We shall term this setup an inertial reference frame, and we restrict to positively oriented coordinate axes at the moment.
It is useful to consider $ t,x,y,z $ as four coordinates $ x^i = (x^O,x^l,x^2,x^3):= (t,x,y,z) $.
Time thus appears, at first in a purely formal manner, as a fourth ('\emph{zeroth}') coordinate.

Our next task is to find the relation between different inertial frames. If $ I $ is \emph{inertial}, then from experience we know that a reference frame $ I $ is again inertial if with respect to $ I $ it is 
\begin{enumerate}
	\item parallely displaced by $ \mathbf{a} $
	\item rotated by $ \mathbf{\alpha} $
	\item moving at constant velocity $ \mathbf{v} $
	\item time delayed by $ a^O $.
\end{enumerate}
Here $ \mathbf{\alpha} $ is the rotation vector and $ a^O $ is the time lag between the clocks attached to the two systems; parallel displacement and rotation refer to Euclidean Geometry, valid by experience in every inertial system.\footnote{One does not, however, obtain new inertial systems by considering systems accelerated against $ I $.}
We are thus looking for the transformation\footnote{Formally, the relation between inertial frames $ I, \bar{I} $ is described by specifying, for each event $ x $, the relation between its coordinates $ x^i $ with respect to $ I $ and its coordinates $ x^{\bar{i}} $ with respect to $ I $.}
\begin{equation}
\label{eq:transf:general_form}
x^{\bar{i}}=f^i(x^k).
\end{equation}
The infinite number of transformations must be restricted to the requirement that straight world lines with respect to $ I $ have to be transformed into straight world lines with respect to $ \bar{I} $ by \eqref{eq:transf:general_form}. It is reasonable also to require that finite coordinate values are always transformed to finite ones: it is then well-known that transformations with these properties are given by \emph{affine transformations}:
\begin{equation}
x^{\bar{i}}=L^i_kx^k +a^i, \qquad i\in \{0,1,2,3\}.
\end{equation}