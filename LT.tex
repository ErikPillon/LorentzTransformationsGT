\section{The Lorentz Transformations}
Consider a number of labs in free flight. Such a lab defines
an inertial system $ I $. Each (pointlike) event may be recorded by noting its coordinates $ x, y, z $ with respect to a rectangular Cartesian coordinate system anchored in $ I $ together with the reading $ t $ of a clock attached to $ 1 $. We shall term this setup an inertial reference frame, and we restrict to positively oriented coordinate axes at the moment.
It is useful to consider $ t,x,y,z $ as four coordinates $ x^i = (x^O,x^l,x^2,x^3):= (t,x,y,z) $.
Time thus appears, at first in a purely formal manner, as a fourth ('\emph{zeroth}') coordinate.

Our next task is to find the relation between different inertial frames. If $ I $ is \emph{inertial}, then from experience we know that a reference frame $ I $ is again inertial if with respect to $ I $ it is 
\begin{enumerate}
	\item parallely displaced by $ \mathbf{a} $
	\item rotated by $ \mathbf{\alpha} $
	\item moving at constant velocity $ \mathbf{v} $
	\item time delayed by $ a^O $.
\end{enumerate}
Here $ \mathbf{\alpha} $ is the rotation vector and $ a^O $ is the time lag between the clocks attached to the two systems; parallel displacement and rotation refer to Euclidean Geometry, valid by experience in every inertial system.\footnote{One does not, however, obtain new inertial systems by considering systems accelerated against $ I $.}
We are thus looking for the transformation\footnote{Formally, the relation between inertial frames $ I, \bar{I} $ is described by specifying, for each event $ x $, the relation between its coordinates $ x^i $ with respect to $ I $ and its coordinates $ x^{\bar{i}} $ with respect to $ I $.}
\begin{equation}
\label{eq:transf:general_form}
x^{\bar{i}}=f^i(x^k).
\end{equation}
The infinite number of transformations must be restricted to the requirement that straight world lines with respect to $ I $ have to be transformed into straight world lines with respect to $ \bar{I} $ by \eqref{eq:transf:general_form}. It is reasonable also to require that finite coordinate values are always transformed to finite ones: it is then well-known that transformations with these properties are given by \emph{affine transformations}:
\begin{equation}
x^{\bar{i}}=L^i_kx^k +a^i, \qquad i\in \{0,1,2,3\}.
\label{eq:transf:general_form2}
\end{equation}
\subsection{Consequences of relativity Principle}
Since there are no restrictions on the space-time translations $ a^i $, we will
consider here only the homogeneous transformations, eq.~\eqref{eq:transf:general_form} with a $ i = 0 $, and take up translations only later. As we have stated, there are no absolute directions and velocities. As a consequence, the relation between $ I $ and $ \bar{I} $, and thus the matrix $ L^i_k $ must be expressible by the \emph{axial vector} $ \mathbf{\alpha} $ describing the relative angular orientation between their spatial axes, together with the \emph{polar vector} $ \mathbf{v} $ of relative velocity.

If there is only a relative rotation between the systems, $ L^i_k $ has to be formed from the rotation vector $ \mathbf{\alpha} $ alone. In this case, the coefficient $ L^i_k $ in eq.~\eqref{eq:transf:general_form2} has the form

\[L^i_k(\mathbf{\alpha}) ={\begin{bmatrix}1&0\\0&R^{\mu}_{\nu}(\mathbf{\alpha})\end{bmatrix}} \]

where $ R^{\mu}_{\nu}(\mathbf{\alpha}) $ is a rotation matrix, which rotates any vector in one sense (active transformation), or equivalently the coordinate frame in the opposite sense (passive transformation). 

However, if the systems differ only by uniform rectilinear relative motion, then
only $ v $ is at our disposal for constructing $ L^i_k $, and the transformation must look like
\[\begin{cases}
x^{\bar{0}}=a(v)x^0+b(v)\mathbf{v}\mathbf{x}\\
\bar{\mathbf{x}}=c(v)\mathbf{x}+\frac{d(v)}{v^2}\mathbf{v}(\mathbf{v}x)+e(v)\mathbf{v}x^0.
\end{cases} \]
A first restriction for the unknown functions $ a(v), b(v), c(v), d(v) $ and $ e(v) $ comes from the condition that the origin of $ I $ be moving with velocity $ \mathbf{v} $ relative to $ \bar{I} $, which means that $ \mathbf{x} = \mathbf{v}x^0 $ must imply $ \bar{x} = 0 $, and this is the case if 
\begin{equation}
c(v)+ d(v) + e(v)=O.
\end{equation}
Further conditions for the unknown functions now follow from the principle of
relativity. Let us exchange the roles of $ I $ and $ \bar{I} $: then $ \bar{I} $ is moving against the latter with velocity $ \mathbf{v} = -\mathbf{v} $.

Plugging the equations obtained by putting $ \mathbf{v} = -\mathbf{v} $ into eq.~\eqref{eq:transf:general_form2}, we will obtain an identity only if
\begin{equation}
C^2=1, \quad a^2-ebv^2=1,\quad e^2-ebv^2=1,\quad e(a+e)=0,\quad b(a+e)=0,
\label{eq:system}
\end{equation}
as is best checked by specializing $ \mathbf = (v, 0, 0)^T $.

The value $ c = -1 $ would correspond to a $ 180° $ rotation contained in \eqref{eq:system} and has to be excluded here. From the third equality of eqs.\eqref{eq:system} we have $ e\neq 0 $, hence $ a + e = 0 $ from the fourth. This satisfies the fifth also, and the second and third become equivalent. Thus we have 
\begin{equation}
b=\frac{1-a^2}{av^2}, \qquad c=1,\qquad d=a-1, e=-a.
\end{equation}

The only yet unknown function $ a( v) $ will finally result from the application of the principle of relativity to three inertial frames $ I, \bar{I}, \bar{\bar{I}} $, where $ \bar{I} $ is moving with $ \mathbf{v} $ against $ I $ and $ \bar{\bar{I}} $ moving with speed $ \mathbf{\bar{w}} $ against $ \bar{I} $. If here $ \mathbf{v} $ and $ \mathbf{\bar{w}} $ are proportional, the relation
between $ \bar{\bar{I}} $ and $ I $ has again to be a pure '\emph{boost}' in the same direction.

Putting v$ \mathbf{v} $ and $ \mathbf{\bar{w}} $ into the 1-directions, the product of the transformations must take the form of 



