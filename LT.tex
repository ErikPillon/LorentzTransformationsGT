\section{The Lorentz Transformations}
Consider a number of labs in free flight. Such a lab defines
an inertial system $ I $. Each (pointlike) event may be recorded by noting its coordinates $ x, y, z $ with respect to a rectangular Cartesian coordinate system anchored in $ I $ together with the reading $ t $ of a clock attached to $ 1 $. We shall term this setup an inertial reference frame, and we restrict to positively oriented coordinate axes at the moment.
It is useful to consider $ t,x,y,z $ as four coordinates $ x^i = (x^O,x^l,x^2,x^3):= (t,x,y,z) $.
Time thus appears, at first in a purely formal manner, as a fourth ('\emph{zeroth}') coordinate.

Our next task is to find the relation between different inertial frames. If $ I $ is \emph{inertial}, then from experience we know that a reference frame $ I $ is again inertial if with respect to $ I $ it is 
\begin{enumerate}
	\item parallely displaced by $ \mathbb{a} $
	\item rotated by $ \mathbb{\alpha} $
	\item moving at constant velocity $ \mathbb{v} $
	\item time delayed by $ a^O $.
\end{enumerate}
