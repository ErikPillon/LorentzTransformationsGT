% !TeX spellcheck = en_US
% !TEX encoding = utf8
\documentclass[a4paper,9pt]{article}
\usepackage[english]{babel}
\usepackage[T1]{fontenc} 
\usepackage[utf8]{inputenc}

\usepackage{amsmath,amssymb}
\usepackage{amsfonts}
\usepackage{mathtools}

\usepackage{amsthm}
\usepackage{esint}
\usepackage{stix}
\usepackage{tikz}
\usepackage{microtype}

%--------------- Theorem Style ------------------
\theoremstyle{plain}
\newtheorem{thm}{Theorem}[section]
\newtheorem{lem}[thm]{Lemma}
\newtheorem{prop}[thm]{Proposition}
\newtheorem*{cor}{Corollary}

\theoremstyle{definition}
\newtheorem{defn}{Definition}[section]
\newtheorem{conj}{Conjecture}[section]
\newtheorem{exmp}{Example}[section]

\theoremstyle{remark}
\newtheorem*{rem}{Remark}
\newtheorem{note}{Note}
%------------End Theorem Style ------------------

\usepackage{palatino} % font figo 
\usepackage{url}
\usepackage{enumerate} %per le liste in lettere romane
\usepackage{empheq} %per evidenziare le formule
\usepackage{xcolor}

%% AMBIENTI MATEMATICI
\newcommand{\R}{\mathbb{R}}
\newcommand{\Rn}{\mathbb{R^n}}
\newcommand{\C}{\mathbb{C}}
\newcommand{\tu}{\tilde{U}}
\renewcommand{\hat}{\widehat}
\renewcommand{\phi}{\varphi}
\newcommand{\diff}{\mathop{}\!d}
\newcommand{\norm}[1]{\left\lvert #1\right\rvert}
\newcommand{\pd}[2]{\frac{\partial #1}{\partial #2}}
\newcommand{\pds}[2]{\frac{\partial^2 #1}{\partial #2^2}}

\usepackage{color}

%% PACCHETTO GEOMETRY
\usepackage{geometry}
\geometry{a4paper,top=3cm,bottom=3cm,left=3.5cm,right=3.5cm,	
	heightrounded,bindingoffset=5mm}
\usepackage{natbib}
% --------------- Hyperref setup ----------------------
\usepackage{hyperref}
\hypersetup{colorlinks=true,
	linkcolor=blue,
	urlcolor=blue,
	citecolor=green,
	anchorcolor = yellow
}
% ------------ end Hyperref setup ----------------------

\frenchspacing 
\title{Lorentz Transformations: a Group Theory approach}
\author{Neil \& Pillon Erik}
% DOCUMENT												
\begin{document}
	
%\begin{flushright}
%	\emph{"So, ultimately, in order to understand nature it may be necessary to have a deeper understanding of mathematical relationships. But the real reason is that the subject is enjoyable, and although we humans cut nature up in different ways, and we have different courses in different departments, such compartmentalization is really artificial, and we should take our intellectual pleasures where we find them."}\\ 
%	
%	\medskip
%	
%	Richard Phillips Feynman (1918,1988)
%\end{flushright}


\maketitle
\begin{flushright}
	\emph{"The fact that mathematics does such a good job\\ of describing the Universe is a mystery that we don't understand.\\ And a debt that we will probably never be able to repay."}\\ 
	
	\medskip
	
	William Thomson, 1$^{st} $ Baron Kelvin
\end{flushright}

\begin{abstract}
In this work we will focus on Lorentz transformations, that are coordinate transformations between two coordinate frames that move at constant velocity relative to each other.We recall that the term "Lorentz transformations" only refers to transformations between inertial frames, usually in the context of special relativity. 	

In this work we will derive these transformations focusing on Group Theory properties. 
\end{abstract}


% MAIN													

\section{Introduction}
In each reference frame, an observer can use a local coordinate system (most exclusively Cartesian coordinates in this context) to measure lengths, and a clock to measure time intervals. An observer is a real or imaginary entity that can take measurements, say humans, or any other living organism—or even robots and computers. An event is something that happens at a point in space at an instant of time, or more formally a point in spacetime. The transformations connect the space and time coordinates of an event as measured by an observer in each frame.
\section{The Lorentz Transformations}
Consider a number of labs in free flight. Such a lab defines
an inertial system $ I $. Each (pointlike) event may be recorded by noting its coordinates $ x, y, z $ with respect to a rectangular Cartesian coordinate system anchored in $ I $ together with the reading $ t $ of a clock attached to $ 1 $. We shall term this setup an inertial reference frame, and we restrict to positively oriented coordinate axes at the moment.
It is useful to consider $ t,x,y,z $ as four coordinates $ x^i = (x^O,x^l,x^2,x^3):= (t,x,y,z) $.
Time thus appears, at first in a purely formal manner, as a fourth ('\emph{zeroth}') coordinate.

Our next task is to find the relation between different inertial frames. If $ I $ is \emph{inertial}, then from experience we know that a reference frame $ I $ is again inertial if with respect to $ I $ it is 
\begin{enumerate}
	\item parallely displaced by $ \mathbb{a} $
	\item rotated by $ \mathbb{\alpha} $
	\item moving at constant velocity $ \mathbb{v} $
	\item time delayed by $ a^O $.
\end{enumerate}

\section{Invariance of the Speed of Light.}
The yet undetermined constant $ K $ has the physical dimension of reciprocal velocity squared. To interpret it we remark that the foundamental relation holds
\[(dx^0)^2+K(d\mathbf{x})^2=(dx^{\bar{0}})^2+K(d\mathbf{\bar{x}})^2 \]
As a consequence, for any motion $ \mathbf{x}=\mathbf{x}(x^0) $ satisfying $ (d\mathbf{x}/dx^O)^2=-1/K $ in one inertial system the analogous relation is true in any other inertial system. Therefore, $ c:= 1/\sqrt{-K} $ plays the role of a uniquely determined \emph{invariant speed}.\footnote{It is an experimental question whether such exists in nature, and if so, what is its value.}

In what follows, we shall most of the time assume performed the rescaling indicated above, and use units where $ c = 1 $-i.e., speeds are expressed as multiples of $ c $.

Then we have
\begin{equation}
K=-1,\qquad a(v)=\frac{1}{\sqrt{1-v^2}}=:\gamma,\qquad \frac{\gamma-1}{v^2}\equiv \frac{\gamma^2}{\gamma+1},
\end{equation}
and eq.~\eqref{eq:semifinal_formulation} becomes the (special) Lorentz transformation ('\emph{Lorentz boost}')
\begin{equation}
\begin{cases}
x^{\bar{0}}=\gamma(x^0-\mathbf{v}\mathbf{x})\\
\bar{\mathbf{x}}=\mathbf{x}+\frac{\gamma^2}{\gamma+1}\mathbf{v}(\mathbf{v}\mathbf{x})-\gamma\mathbf{v}x^0.
\end{cases}
\end{equation}
By composing space-time translations, space rotations and Lorentz boosts in various ways we get more complicated transformations. Homogeneous ones will be called \emph{(general) Lorentz transformations}, inhomogeneous ones will be called \emph{Poincar\'e transformations}.







% APPENDICES											
\appendix


% REFERENCES											
\nocite{sexl2012relativity}
\thispagestyle{empty}
\addcontentsline{toc}{section}{References}

\bibliographystyle{plain} 
\bibliography{bib}
% For citing with natbib use \citet and \citep commands
% \citet cites as a part of the sentence in form Author (year)
% \citep cites in parentheses in form (Author, year)
\end{document}
