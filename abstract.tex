\begin{abstract}
In this work we will focus on Lorentz transformations, that are coordinate transformations between two coordinate frames that move at constant velocity relative to each other.We recall that the term "Lorentz transformations" only refers to transformations between inertial frames, usually in the context of special relativity. They supersede the \emph{Galilean transformation} of Newtonian mechanics, which assumes an absolute space and time. The Galilean transformation is a good approximation only at relative speeds much smaller than the speed of light. Lorentz transformations have a number of unintuitive features that do not appear in Galilean transformations.	

In this work we will derive these transformations focusing on Group Theory properties. 
\end{abstract}
